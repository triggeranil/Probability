\documentclass[journal,12pt,two column]{IEEEtran}
%\usepackage{setspace}
\usepackage{listings}
\usepackage{amssymb}
\usepackage[cmex10]{amsmath}
\usepackage{amsthm}
\usepackage[export]{adjustbox}
\usepackage{bm}
\def\inputGnumericTable{} 

\usepackage[latin1]{inputenc}                                 
\usepackage{color}                                            
\usepackage{array} 
\usepackage{longtable} 
\usepackage{calc}                                             
\usepackage{multirow}                                         
\usepackage{hhline}                                           
\usepackage{ifthen}  
\usepackage{mathtools}
\usepackage{tikz}
\usepackage{listings}
\usepackage{color}                                            %%
\usepackage{array}                                            %%
\usepackage{caption} 
\usepackage{graphicx}

\title{A1110 Assignment 1 \\ 11.16.3.8}
\author{Megavath Anil Nayak \\ AI22BTECH11019 \\
        \thanks{*The student is from Department of Chemical Engineering, Indian Institute of Technology, Hydeabad. e-mail: ai22btech11019iith.ac.in}}
\providecommand{\pr}[1]{\ensuremath{\Pr\left(#1\right)}}
\providecommand{\qfunc}[1]{\ensuremath{Q\left(#1\right)}}
\providecommand{\sbrak}[1]{\ensuremath{{}\left[#1\right]}}
\providecommand{\lsbrak}[1]{\ensuremath{{}\left[#1\right.}}
\providecommand{\rsbrak}[1]{\ensuremath{{}\left.#1\right]}}
\providecommand{\brak}[1]{\ensuremath{\left(#1\right)}}
\providecommand{\lbrak}[1]{\ensuremath{\left(#1\right.}}
\providecommand{\rbrak}[1]{\ensuremath{\left.#1\right)}}
\providecommand{\cbrak}[1]{\ensuremath{\left\{#1\right\}}}
\providecommand{\lcbrak}[1]{\ensuremath{\left\{#1\right.}}
\providecommand{\rcbrak}[1]{\ensuremath{\left.#1\right\}}}
\newcommand*{\permcomb}[4][0mu]{{{}^{#3}\mkern#1#2_{#4}}}
\newcommand*{\perm}[1][-3mu]{\permcomb[#1]{P}}
\newcommand*{\comb}[1][-1mu]{\permcomb[#1]{C}}
\renewcommand{\thetable}{\arabic{table}} 
\newcommand{\question}{\noindent \textbf{Question: }}	
\newcommand{\solution}{\noindent \textbf{Solution: }}
\begin{document}
\maketitle
\question: 8 : Three coins are tossed once. Find the probability of getting:\\
$(i)$ 3 heads  $(ii)$2 heads   $(iii)$ Atleast 2 heads \\
$(iv)$ Atmost 2 heads  $(v)$ No head  $(vi)$ 3 tails  \\
$(vii)$ Exactly 2 tails  $(viii)$ No tail  $(ix)$ Atmost 2 tails \\
\solution : Let $X$ be a random variable such that:
    \begin{align*} 
         X \coloneqq Number of heads \\\\
         X 
         \begin{cases}
            0, \\
            1, \\
            2, \\
            3 \\
         \end{cases}
    \end{align*}
    Let $S$ denote sample space of possible outcomes when the coins are tossed. \\
    Then $\mid S \mid = 2^3 = 8$ \\\\
    (i) \pr{X  = 3} = $\frac{1}{8}$ \\\\
    (ii) \pr{X = 2} = $\frac{^3C_2}{8} = \frac{3}{8}$ \\\\
    (iii) \pr{X \geq 2} = $\frac{^3C_3 + ^3C_2}{8} = \frac{4}{8}$ \\\\
    (iv) \pr{X \leq 2} = $\frac{8 - ^3C_3}{8} = \frac{7}{8}$ \\\\
    (v) \pr{X = 0} = $\frac{^3C_0}{8} = \frac{1}{8}$ \\\\
    (vi) \pr{X = 0} = $\frac{^3C_0}{8} = \frac{1}{8}    (\because 3 tails \equiv 0 heads)$ \\\\
    (vii) \pr{X = 1} = $\frac{^3C_1}{8} = \frac{3}{8}   (\because 2 tails \equiv 1 head)$ \\\\
    (viii) \pr{X = 3} = $\frac{^3C_3}{8} = \frac{1}{8}   (\because 0 tails \equiv 3 heads)$ \\\\
    (ix) \pr{X \geq 1} = $\frac{8 - ^3C_0}{8} = \frac{7}{8}$    (\because atmost 2 tails \equiv atleast 1 head) \\\\ 
\end{document}
